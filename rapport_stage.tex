\documentclass[12pt]{article}
\usepackage[utf8]{inputenc}
\usepackage[T1]{fontenc}
\usepackage[french]{babel}
\usepackage{graphicx}
\begin{document}
\title{Rapport de stage}
\maketitle


Le  sujet du stage concerne l'exploitation des données (lectures)
produites par les technologies de séquençage de 3e génération.

Contrairement aux précédentes générations qui fournissaient des lectures
de longueur constante et petite (au plus 400 bp), les technologies de 3e
génération produisent une distribution de longueurs avec une médiane à
environ 12 kbp et une queue de distribution qui va jusqu'à 60 kbp pour les
lectures les plus longues.

Dans un premier temps, j'aimerais vérifier l'idée que les distributions ainsi obtenues sont des distributions bêta (cas classique de la distribution
des longueurs des morceaux d'un bâton qui a été découpé en N segments
aléatoirement). Dans le cas du séquençage, N est sans doute aussi une
variable aléatoire avec une distribution qui reste à déterminer.
Nous avons dans un premier temps vérifier que les tailles de lectures suivent une loi de distribution bêta.

\includegraphics[scale=0.5]{taille_reads_minion.png} 

C'est le graphique de l'histogramme des tailles de reads.
On sait que cette distribution de tailles suit une loi de bêta de paramètres 1 et n.
On peut le montrer théoriquement :
En effet, cf statistique d'ordre. et page 3 de Pycke

On fit cet histogramme avec distribution bêta(1,N) où N est le nombre de lectures.
On coupe aléatoirement un "bâton" en N morceaux aléatoirement.
On prend un "bâton" de taille 1 000 000 que l'on coupe en 60 000 morceaux aléatoirement. Nous obtenons l'histogramme suivants : \\
\includegraphics[scale=0.5]{hist.png}

Pour vérifier que cette distribution suit une loi bêta de paramètres (1, 60 000), nous allons normaliser les tailles obtenues en les divisant par 1 000 000.\\
\includegraphics[scale=0.5]{histbeta.png}\\
La courbe rouge correspond à la densité de la loi bêta (1,60 000).\\
On a estimer les paramètres de la loi bêta en utilisant la méthode du maximum de vraisemblance. En effet, nous avons minimiser la log vraisemblance négative et nous avons trouvé comme paramètres : 
alpha = 1.1752289930435111 et bêta = 70512.1437318023
Nous avons également estimé ces paramètres en utilisant la méthode des moments et on trouve : 
alpha = 1.0575435001952862 et bêta = 63451.552468207636

\includegraphics[scale=0.5]{histbetamv.png}\\
Pour déterminer la variable aléatoire N on a tout d'abord testé plusieurs lois : loi normale, uniforme, poisson. 
Avec N=65603, correspondant à la moyenne de chacune de ces lois.
\\
\includegraphics[scale=0.5]{normale.png}
\includegraphics[scale=0.5]{uniforme.png}\\
\includegraphics[scale=0.5]{poisson.png}

On sait par ailleurs que lors du séquençage avec Minion, les plus petits fragments ont plus de chance "de passer" que les gros.
On peut alors établir un seuil. 




\end{document}